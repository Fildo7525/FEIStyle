% !TEX root = ../main.tex

\section{Využitie}
\label{sec:vyuzitie}

Robot, ktorý sme dostali na tento projekt bol povodne ovládateľný len pomocou JSON reťazcov. My sme na ňom museli spraviť
par zmien aby sme vedeli počítať odometriu a tým pádom mali spätnú väzbu. Pri odometrii sme čelili pár prekážkam, ktoré
sa nám podarilo z časti odstrániť. Skončili sme s ovládačom nenaprogramovaným v jazyku C++ v prostredí ROS2.

Ďalšou otázkou môže byť ako sa tento ovaliac môže využiť? Odpoveď na túto otázku nie je jedna ale rovno viacero.
Prostredie ROS2 nám otvára mnoho ďalších možnosti ako posielať príkazy na robot. Ich implementácia je nastokrát už
dokončená a my môžeme tieto kódy len jednoducho pripojiť k programu. Možnosťami sú napríklad:

\begin{itemize}
	\item \textbf{Manuálne ovládanie} Ovládanie pomocou
		\begin{itemize}
			\item \textbf{klávesnice} použitie uzlu \textit{teleop\_twist\_keyboard}
			\item \textbf{joystick} použitie uzlu \textit{teleop\_twist\_joy}
			\item \textbf{príkazov} skripty poprípade priame zadávanie príkazov do konzoly
		\end{itemize}
	\item \textbf{Automatické ovládanie}
\end{itemize}

