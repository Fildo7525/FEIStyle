% !TEX root = ../main.tex

\section{Využitie}
\label{sec:vyuzitie}

Robot, ktorý sme dostali na~tento projekt bol povodne ovládateľný len pomocou JSON reťazcov. My sme na~ňom museli spraviť
par zmien aby sme vedeli počítať odometriu a~tým pádom mali spätnú väzbu. Pri~odometrii sme čelili pár prekážkam, ktoré
sa nám podarilo z~časti odstrániť. Skončili sme s~ovládačom nenaprogramovaným v~jazyku C++ v~prostredí ROS2.

Ďalšou otázkou môže byť ako sa tento ovaliac môže využiť? Odpoveď na~túto otázku nie je jedna ale~rovno viacero.
Prostredie ROS2 nám otvára mnoho ďalších možnosti ako posielať príkazy na~robot. Ich implementácia je nastokrát
už~dokončená a~my môžeme tieto kódy len jednoducho pripojiť k~programu. Možnosťami sú~napríklad:

\begin{itemize}
	\item \textbf{Manuálne ovládanie} Ovládanie pomocou
		\begin{itemize}
			\item \textbf{klávesnice} použitie uzlu \textit{teleop\_twist\_keyboard}
			\item \textbf{joystick} použitie uzlu \textit{teleop\_twist\_joy}
			\item \textbf{príkazov} skripty poprípade priame zadávanie príkazov do~konzoly
		\end{itemize}
	\item \textbf{Automatické ovládanie}
\end{itemize}

\subsection{Joystick}
\label{subsec:joystick}

Joystick je ovládač obyčajne používaný pri~herných konzolách. My sme ho použili na~manuálne ovládanie robota. Ako bolo
spomenuté vyššie (Kapitola~\ref{sec:vyuzitie}) joystick je možné použiť s~uzlom \textit{teleop\_twist\_joy}. Tento uzol
sprostredkuje využitie hardware-u ovládača. Pohyby vykonané na~ovládači sa transformujú do~typu správy
\textit{geometry\_msgs/Twist} a~publikujú sa priamo na~tému \textit{cmd\_vel}. Táto téma je priamo pripojená
na~náš uzol \textit{BlackMetal} (Kapitola~\ref{subsec:nodes}). Prijatú správu nami vytvorený ovládač spracuje a~pošle
ju robotu. Ovládač je možné spustiť s~viacerými tipmi ovládačov sú~to~\textit{atk3}, \textit{ps3-holonomic},
\textit{ps3}, \textit{xbox}, \textit{xd3} \cite{teleopjoy}. Po~pripojení ovládača k~nášmu programu a~prepojení
programu s~robotom sme mohli začať s~jeho ovládaním.

Pri~používaní ovládača nastáva jeden problém a~to~je, že~pri~hocijakom pohybe joysitcku sa pošle požiadavka na~zmenu
rýchlosti robota. To~spôsobuje zahltenie odosielacej rady. To~má za~dôsledok zlyhanie odometrie. Môže sa stať,
že~odometria nezlyhá a~začne posielať príkazy robotu. Keďže je vstavané oneskorenie do~posielania správ robota,
tak~na~poslanie každej správy z~odometrie potrebujeme 100 milisekúnd. Ak~pridáme do~toho ešte neustále posielanie správ
z~joysitcku, tak sa môže stať, že~robot začne sekať. Ako vieme tento problém vyriešiť je zväčšiť interval posielania
správ odometrie. To~bude mať za~dôsledok nepresnejšie určenie polohy.

