% !TEX root = ../main.tex

Hlavným cieľom bakalárskej prace bolo vytvorenie ovládača pre mobilného robota v prostredí ROS2. Hlavni úlohu
ovládača tvoril uzol, ktorý zabezpečoval prijímanie a spracovanie dát od užívateľa. Jeho ďalšou úlohou bolo vytvorenie
klienta a odometrie. Klient zaisťoval výmenu požiadavok a odpovedi so serverom, ktorý sa nachádzal na robote. Odometria
na druhú stranu zabezpečovala spracovanie prijatých sprav od robota a ich následnú publikáciu na tému /odom. Pre
validáciu presnosti odometrie sme spravili s robotom test ako presne vypočíta dráhu štvorca, ktorú prejde robot.
Ovládač je možné použiť na ovládanie robota pomocou preddefinovaných uzlov alebo pomocou vlastného uzla, ktorý
bude publikovať spravy na tému /cmd\_vel. Vytvorený ovládač sme otestovali s uzlom \textit{teleop\_twist\_joy}.

Pri prijímaní dát z robota sme zistili, že enkódery posielajú Zašumený signál. Tento problém sme vyriešili implementáciu
kvadratického dolno-priepustného filtra. Dolno-priepustný filter zabezpečil stabilitu dát, z ktorých sme následne počítali
aktuálnu polohu robota.

Nami navrhnutý ovládač je len jednou z možností ako navrhnúť tento ovládač. My sme boli obmedzení tým, že na robote,
ktorý sme ovládali sa nenachádzal robotický operačný systém. Ak by sa na robote tento systém nachádzal, tak by sme mohli
použiť už existujúci balíček \textit{ros2\_control}. Ten bol navrhnutý a naprogramovaný skúsenejšími programátormi.
Jeho efektivita pri ovládaní robota by bola preto lepšia ako pri našom ovládači.

