V prvej časti bakalárskej prace bolo našou úlohou zoznámiť sa s robotom a neimplementovať ovládač na neho pomocou ROSu.
V stave v akom sme ho dostali sme museli opraviť niektoré softvérové chyby na robote. Tie sa týkali hlavne komunikácie s robotom.
V tejto práci sme sa dostali do stavu, kedy vieme robotu poslať, akou rýchlosťou sa majú hýbať jednotlive kolesa a on nám dáva spätnú väzbu,
že akou rýchlosťou naozaj ide. To ale ešte nefunguje, tak ako má kvôli zašumenému signálu. V druhej časti bakalárskej prace by sme chceli opraviť
alebo aspoň zredukovať toto zašumenie. Poprípade nájsť iný spôsob na kontrolovanie pozície robota v priestore.

