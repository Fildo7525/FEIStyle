% !TEX root = ../main.tex

Hlavným cieľom bakalárskej prace bolo vytvorenie ovládača pre~mobilného robota v~prostredí ROS2. Aby sme mohli začať
s~programovaním tohto ovládača museli sme opraviť pár chýb, ktoré sa na~robte vyskytli. Boli to~chyba v~systéme
(\ref{subsec:brokenSystem}), oneskorenie komunikácie robota (\ref{subsec:communicationDelay}) a~nesprávna spätná väzba
robota (\ref{subsec:wrongFeedback}).

Hlavnú úlohu ovládača tvoril uzol, ktorý zabezpečoval prijímanie a~spracovanie dát od~užívateľa (\ref{subsec:nodes}).
Jeho ďalšou úlohou bolo vytvorenie klienta a~odometrie. Klient zaisťoval výmenu požiadavok a~odpovedi so~serverom,
ktorý sa nachádzal na~robote. Odometria na~druhú stranu zabezpečovala spracovanie prijatých sprav od~robota a~ich
následnú publikáciu na~tému /odom. Pre~validáciu presnosti odometrie sme spravili s~robotom test ako presne vypočíta
dráhu štvorca, ktorú prejde robot. Ovládač je možné použiť na~ovládanie robota pomocou preddefinovaných uzlov alebo
pomocou vlastného uzla, ktorý bude publikovať spravy na~tému /cmd\_vel. Vytvorený ovládač sme otestovali s~uzlom
\textit{teleop\_twist\_joy}.

Pri~prijímaní dát z~robota sme zistili, že~enkódery posielajú zašumený signál. Tento problém sme vyriešili implementáciu
kvadratického dolno-priepustného filtra. Dolno-priepustný filter zabezpečil stabilitu dát, z~ktorých sme následne počítali
aktuálnu polohu robota.

Nami navrhnutý ovládač je len jednou z~možností ako navrhnúť tento ovládač. Pre~budúce ovládače robotov by som navrhoval
použitie druhej verzie robotického operačného systému. Jeho použitie umožňuje jednoduchšie a~bezpečnejšie programovanie,
pričom zároveň poskytuje možnosť jednoduchého rozširovania ovládača o~ďalšie uzly.

