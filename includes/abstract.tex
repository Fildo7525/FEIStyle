% !TEX root = ../main.tex
V tejto bakalárskej práci sa zaoberáme riadením mobilného robota v prostredí ROS2. Robot, ktorý budeme využívať
je vybavený diferenciálnym podvozkom a dvoma opornými všesmerovými kolesami. Obsahuje dva enkódery pre pravé
a ľavé koleso, ktoré merajú rýchlosti otáčania kolies v impulzoch za sekundu. Jeho ovládanie a komunikácia
s ním prebieha cez TCP/IP spojenie. Pri komunikácii s robotom sa posielajú spravy v štruktúre JSON.
Výsledným cieľom tejto bakalárskej práce je vytvoriť ovládač robota, ktorý bude sprístupňovať ovládanie robota
v prostredí ROS2. V prvej kapitole sa venujeme analýze pôvodného stavu robota a jeho komunikačnému rozhraniu.
V druhej kapitole popisujeme návrh implementácie ovládača robota v prostredí ROS2. Tretiu kapitolu sme venovali
popísaniu zlepšení, ktoré sme vykonali na robote. Štvrtá kapitola sa venuje samotnej implementácii ovládača robota.
V predposlednej kapitole opisujeme ako sme prekonali problémy, ktoré vznikli pri komunikácii s robotom. V poslednej
kapitole opisujeme viacero využití ovládača robota v prostredí ROS2.
