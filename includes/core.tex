% !TEX root = ../main.tex
\section{Návrh implementácie riadenia}
\label{sec:technologies}

Našou úlohou v~tejto bakalárskej práci je naprogramovať a~otestovať nami vytvorený ovládač pre~mobilného
robota. Na~vyber máme viacero technológii, ktoré môžeme použiť. Na~vytvorenie ovládača budeme potrebovať
operačný systém na~počítači, ktorý budeme použivať na~programovanie a~testovanie. Na~výber máme dve možnosti.
Menovite to~sú~operačný systém \textit{Windows} alebo operačný systém \textit{Ubuntu}, ktorý je založený
na~jadre Linux. Ďalšiu vec, ktorú budeme potrebovať je programovací jazyk. Na~výber máme z~veľmi veľa
možností a~to~hlavne \textit{C}, \textit{C++}, \textit{Python}, \textit{Java} alebo \textit{Matlab}.
Keďže, zo~zadania vypláva, že~máme použiť Robotický operačný systém, tak~máme na~výber dve možnosti.
Sú~to~Robotický operačný systém prvej verzie (\textit{ROS1}) a~Robotický operačný systém druhej verzie
(\textit{ROS2}). Na~vytvorenie ovládača môžeme využiť aj~už~naprogramované baličky. Jednou z~možností
je \textit{ros\_control} pre~Robotický operačný systém prvej verzie a~\textit{ros2\_control} pre~Robotický
operačný systém druhej verzie. Samozrejme na~implementáciu a~otestovanie potrebujeme samotného robota.
Pred tým ako začneme preberať spomenutý ovládač a~robota, ktorého budeme riadiť, tak~si predstavíme
technológie, ktoré budeme používať pri~vytváraní ovládača.

\subsection{Operačný systém}
\label{sec:operating_system}

Na~začiatku našej práce sme sa museli rozhodnúť, či~budeme písať program na~náš ovládač v~operačnom systéme Windows 11 alebo
v~operačnom systéme Ubuntu, ktorý je založený na~jadre Linux. Ako programátori sme už~dlhšie pracovali s~operačným systémom
Ubuntu a~poznáme aj~jeho výhody, aj~nevýhody. To~isté platí aj~pre~operačný systém Windows. Toto sú~fakty, ktoré sme zvažovali
pri~výbere operačného systému, ktorý sme používali počas celej bakalárskej práce.

\subsubsection{Windows}
\label{sec:windows}

Windows je najpoužívanejším operačným systémom na~počítače na~svete. Je to~hlavne preto, že~je jednoduchý na~používanie
pre~netechnicky založených užívateľov. Windows ale~nie je bezpečným operačným systémom a~taktiež aj~jeho stabilita a~rýchlosť
nie sú~na~najvyššej úrovni. Tento operačný systém používajú lúdia hlavne na~voľnočasové aktivity ako je hranie hier
alebo v~práci ako napríklad v~kanceláriách na~úpravu tabuliek. Pre~nás ako programátorov tento operačný systém nie je
výhodný. Nastavovanie prostredia pre~programovanie je v~tomto operačnom systéme oveľa náročnejšie ako v~operačnom systéme
s~Linuxovým jadrom. Podpora prekladačov programov (z~anglického compiler) je v~tomto operačnom systéme oveľa horšia
ako už~v~spomenutom Linuxe.

\subsubsection{Ubuntu}
\label{sec:ubuntu}

Ubuntu je stabilný a~bezpečný operačný systém, ktorý je založený na~Linuxe. Väčšina ľudí hovorí o~Linuxe ako o~operačnom systéme. Pravdou
to~ale~nie je. Linux je totižto len \textbf{kernel} (jadro) operačného systému. Kernel je hlavný program, ktorý sa stará o~správu hardvéru,
riadenie procesov a~správu pamäte. Zároveň poskytuje možnosť operačnému systému komunikovať s~hardvérom a~poskytuje jednotlivým programom
prístup k~hardvéru a~jeho častiam~\cite{chatgpt}.

Linux je používaný skoro vo~všetkých mobilných zariadeniach, serveroch, cloudoch (dátové servery, na~ktoré sa pripájame cez~internet)
a~ďalších elektronických zariadeniach, ktoré potrebujú rýchlosť, stabilitu a~bezpečnosť. Je taktiež používaný vo~všetkých mobilných
robotoch. Nad~operačným systémom Windows má niekoľko výhod.

\begin{itemize}
	\item \textbf{Open source} Linux je open source jadro pre~operačný systém, čo znamená, že~kód,\\
		ktorý tvorí jeho základ, je k~dispozícii pre~všetkých zadarmo a~každý môže prispieť
		k~jeho vylepšeniu. Tento otvorený prístup umožňuje programátorom prispôsobiť si Linux
		pre~svoje potreby a~vytvárať programy, ktoré sú~optimálne pre~ich prácu. Na~Linuxe
		sú~založené aj~operačné systémy reálneho času alebo softvére pre~smart mobilné telefóny
		ako je napríklad operačný systém Android.

	\item \textbf{Programovacie nástroje} Linux obsahuje mnoho vynikajúcich programovacích nástrojov, ktoré sú~k~dispozícii
		zadarmo a~sú~široko používané programátormi. Napríklad k~dispozícii v~Linuxe sú~(GCC) \acrlong{GCC} (Kolekcia GNU prekladačov)
		(GNU je skratka pre~\acrlong{GNU}) a~GNU Debugger (GDB) sú~vynikajúce nástroje pre~C/C++ programátorov.

	\item \textbf{Bezpečnosť} Linux má reputáciu bezpečnejšieho operačného systému v~porovnaní s~operačným systémom Windows.
		V~Linuxe je ťažšie infikovať sa škodlivými programami ako je~vírus alebo malware, pretože programy majú obmedzené
		oprávnenie a~prístup k~ďalším súborom. Systém je navrhnutý tak, aby bol odolný voči útokom.

	\item \textbf{Stabilita} Linux má tendenciu byť stabilnejší než Windows, pretože nie je závislý na~ovládačoch a~softvéri
		od~tretích strán. Linux poskytuje jednotný prístup k~správe a~aktualizácii softvéru. Pri~vylepšovaní systému nie je
		za~potreby reštartovať počítač alebo server čo má za~dôsledok zvýšenie spoľahlivosti zariadenia.

	\item \textbf{Flexibilita} Linux je veľmi flexibilný operačný systém, ktorý umožňuje programátorom prispôsobiť si svoje pracovné
		prostredie a~používať programy, ktoré najlepšie vyhovujú ich potrebám. Linux dovoľuje upraviť samotné jadro
		operačného systému a~vybrať si z~neho len tie časti, ktoré užívateľ potrebuje. Tie, ktoré nepotrebuje sa
		v~systéme nenachádzajú. Týmto spôsobom sa zvyšuje rýchlosť systému a~zabezpečuje sa jeho bezpečnosť.

	\item \textbf{Podpora a~komunita} Linux má veľkú a~aktívnu komunitu. Ta obsahuje ako aj~programatorov tak~aj
		netechnicky založených ľudí. Táto komunita poskytuje podporu a~rieši problémy. Vďaka tomu, že~Linux je
		open source, mnoho ľudí prispieva k~jeho vylepšeniu a~vytvára nové programy, čo znamená, že~je vždy
		niečo nové na~objavovanie.

	\item \textbf{Komplexnosť} Napriek všetkým týmto výhodám má Linux aj~svoje nevýhody. Najväčšou nevýhodou, ktorú je treba
		spomenúť je komplexnosť operačného systému. Linux je komplexnejší ako operačný systém Windows, hlavne kvôli svojej
		flexibilnosti. Tým, že~si vieme nastaviť vlastné prostredie, typ systému správy súborov, programy na~sťahovanie,
		publikovanie a~správy balíčkov. Ľudia, ktorí nevedia tento systém opraviť, často siahajú po~preinštalovaní
		systému.
\end{itemize}

\noindent Vďaka svojim výhodám sme sa rozhodli použiť operačný systém Ubuntu verzie 22.04.

\subsection{Programovací jazyk a~jeho prostredie}
\label{sec:programovaci_jazyk}

Medzi programovacie jazy, ktoré by sme mohli použiť patria: \textit{C++}, \textit{C}, \textit{Python},
\textit{Matlab}, \textit{Java}. Aby sme sa rozhodli, ktorý jazyk použijeme pri~vytváraní ovládača, tak~sme si spísali
ich výhody a~nevýhody.

\begin{itemize}
	\item \textbf{C a~C ++}: C a~C ++ sú~hlavné jazyky používané pri~vývoji ROS aplikácií. Sú~to~výkonné a~efektívne jazyky,
		ktoré možno použiť na~nízko úrovňové programovanie, ako sú~riadiace slučky, ovládače a~systémové programovanie.
		C++ je najčastejšie používaný jazyk v~ROS2.

	\item \textbf{Java}: Java je ďalší jazyk, ktorý sa môže použiť na~implementáciu aplikácií v~Robotickom operačnom
		systéme. Je menej bežné používaný ako jazyk C++, ale~má výhodu, a~to~takú, že~je prenositeľný jazyk, ktorý môže
		byť spustený na~akejkoľvek platforme, ktorá podporuje JVM (\acrlong{JVM}) (Virtuálny stroj podporujúci Javu).
		Java je vhodná pre~vývoj vysoko úrovňových komponentov robotického systému, ako sú~GUI (z~anglického \acrlong{GUI})
		(užívateľské rozhranie) a~sieťová komunikácia.

	\item \textbf{Python}: Python je populárny jazyk pre~vývoj ROS2 kvôli svojej čitateľnosti a~jednoduchosti použitia.
		Je vhodný pre~vývoj vysoko úrovňových komponentov robotického systému, ako sú~algoritmy na~rozhodovanie,
		spracovanie dát a~vizualizácia.

	\item \textbf{Matlab}: Matlab je vysoko úrovňový programovací jazyk a~prostredie používané vo~vedeckom prostredí,
		analýze dát a~vizualizácii. Je to~výkonný jazyk pre~numerické výpočty a~často sa používa na~simulovanie
		a~testovanie algoritmov pre~roboty. Avšak, nie je bežne používaný v~ROS2 vývoji. Na~použivanie tohto rozhrania
		je spravený modul \textit{ROS Toolbox}, ktorý zabezpečuje jednoduchú implementáciu uzlov a~ich komunikácie.

	\item \textbf{Výkon}: C a~C++ poskytujú najlepší výkon pre~ROS aplikácie kvôli ich nízkoúrovňovej implementácii a~priamemu
		prístupu k~pamäti. Python, Matlab a~Java sú~pomalšie ako C a~C ++ kvôli ich interpretácii.

	\item \textbf{Čas vývoja}: Python a~Java sa vyvíjajú rýchlejšie ako C a~C++ kvôli ich jednoduchosti a~jednoduchosti
		použitia. Matlab môže byť tiež rýchlejší pri~vývoji pre~určité úlohy vďaka svojim zabudovaným knižniciam a~nástrojom.
		Medzi tieto nástroje patri napríklad \textit{Simunlink} alebo \textit{Stateflow}.

	\item \textbf{Kompatibilita}: Všetky uvedené jazyky môžu byť použité s~Robotickým operačnom systémom a~môžu
		komunikovať s~robotom pomocou TCP/IP komunikácie.
\end{itemize}

Na~základe vyššie spomenutých rozdielov sme sa rozhodli použiť programovací jazyk \textit{C++}. Toto rozhodnutie sme spravili
hlavne kvôli jeho možnostiam v~implementácii tried, výkonu a~podpore v~komunite.

\subsection{Robotický operačný systém}
\label{sec:ros}

Robotický operačný systém (ROS) (\acrlong{ROS}) je súbor voľne dostupných softvérových knižníc a~nástrojov, ktoré vytvárajú
vhodné podmienky pre~programátorov na~písanie aplikácií pre~mnohé druhy robotov. ROS má dve verzie. Vo~všeobecnosti sa stretneme
s~tým, že~pod názvom ROS1 alebo ROS sa myslí ROS verzie 1. Pod názvom ROS2 sa myslí ROS verzie 2. Aby nenastali nejasnosti
budeme v~tomto dokumente označovať ROS verzie 1 ako ROS1 a~ROS verzie 2 ako ROS2. V~prípade, keď budme hovoriť o~spoločných
vlastnostiach a~funkcionalitách, ROS1 a~ROS2 budeme označovať dokopy ako ROS.

ROS je open source, čo znamená, že~ide o~softvér uvoľnený pod licenciou, kde~má používateľ práva na~použitie,
štúdium, zmenu a~redistribúciu. Mnohé open source licencie však určujú určité obmedzenia na~tieto práva pre~tento softvér,
ale~v~podstate môžeme predpokladať tieto práva. Najbežnejšie licencie pre~ROS2 softvérové balíčky sú~Apache 2 a~BSD,
hoci vývojári majú slobodu používať aj~iné \cite{ROS2book}. Využitia Robotického operačného systému v~reálnom svete sa môžu
objaviť v~automobiloch, mobilných robotoch, dronoch alebo manipulátoroch.

\subsubsection{Základné pojmy}
\label{sec:zakladne_pojmy}

Komunikácia v~ROSe je zabezpečená cez~IPC (\acrlong{IPC}), TCP/IP a~UDP/IP komunikáciou pomocou troch zakladacích metód:
\textbf{témy} (Topics), \textbf{služby} (Services) a~\textbf{akcie} (Actions).
Všetky tieto metódy slúžia na~posielanie správ na~synchronizáciu vykonávania programu a~komunikáciu jednotlivých častí programu.
Na~to~aby sme ich vedeli správne využiť musíme poznať ich základnú implementáciu a~funkcionalitu. Správanie týchto metód je nasledovné.

\subsubsection{Témy}
\label{sec:topic}

	\textbf {Témy} sú~sprostredkované pomocou IPC, medzi procesová komunikácia z~anglického \acrlong{IPC},
	alebo TCP/IP komunikácie . Je to~najjednoduchší spôsob komunikácie. Komunikáciu cez~témy si vieme prirovnať k~UDP/IP
	protokolu, s~tým že~nemusia prebiehať cez~sieť. Definujeme si jedného poskytovateľa (publishers) a~jedného
	alebo viacerých príjemcov (subscribers). Medzi týmito dvoma alebo viacerými účastníkmi sa následne
	posielajú správy (messages), ktoré sme si dopredu definovali. Parametre a~obsah týchto správ je
	definovaný priamo v~téme. Nové témy si vieme sami vytvoriť. V~prípade, že~chceme vytvoriť novú tému, musíme
	si vytvoriť nový balíček, v~ktorom si spravíme súbor s~príponou \textit{.msg}.

	\begin{figure}[!htbp]
		\centering
		\includegraphics[width=8cm]{img/topicsExplanation.png}
		\caption{Vizualizácia témy v~ROSe~\cite{RosDoc}}
		\label{fig:topics}
	\end{figure}

\clearpage

\subsubsection{Služby}
\label{sec:services}

	\textbf {Služby} sú~sprostredkované pomocou TCP/IP protokolu. Poskytujú nám rovnaký spôsob komunikácie ako témy, až na~to, že~sa správy
	medzi servisom a~klientom posielajú cez~LAN (\acrlong{LAN}). Tieto správy sa posielajú oboma smermi. Služby sa využívajú pri~komunikácii
	medzi viacerými zariadeniami. Jedno zariadenie posiela požiadavku (request) a~druhe zariadenie túto požiadavku prime
	a~spracuje. Po~spracovaní pošle naspäť odpoveď (response). Odpoveď môže obsahovať jednoduchú potvrdenie, alebo inú
	komplexnejšiu štruktúru. Nové služby si vieme sami vytvoriť. V~prípade, že~chceme vytvoriť novú službu, musíme
	si vytvoriť nový balíček, v~ktorom si spravíme súbor s~príponou \textit{.srv}.

	\begin{figure}[!htbp]
		\centering
		\includegraphics[width=8cm]{img/serviceExplanation.png}
		\caption{Vizualizácia služby v~ROSe~\cite{RosDoc}}
		\label{fig:service}
	\end{figure}

\subsubsection{Akcie}
\label{sec:actions}
	\textbf {Akcie} sú~taktiež sprostredkované TCP/IP protokolom. Sú~najzložitejším spôsobom
	komunikácie. Tento spôsob bol pridaný do~ROS1 až neskôr. V~druhej verzii ROSu je tento
	typ komunikácie medzi troma základnými formami komunikácie uzlov. Sú~založené na~službách
	a~prebiehajú asynchrónne~\cite{ROS2book}. Máju 3 stavy viď Obr.~\ref{fig:action}.
	Najprv pošle klient serveru, akú akciu má vykonať, server mu potvrdí, že~túto požiadavku
	dostal. Server začne následne vykonávať danú akciu a~posielať klientovi priebežné správy
	o~priebehu vykonávania žiadanej úlohy. Keď server skončí, pošle klientovi výsledok akcie
	a~klient mu obratom potvrdí obdržanie výsledku. Nové akcie si vieme sami vytvoriť.
	V~prípade, že~chceme vytvoriť novú akciu, musíme si vytvoriť nový balíček, v~ktorom
	si spravíme súbor s~príponou \textit{.action}.

\clearpage

	\begin{figure}[!htbp]
		\centering
		\includegraphics[width=8cm]{img/actionExplanation.png}
		\caption{Vizualizácia akcie v~ROSe~\cite{RosDoc}}
		\label{fig:action}
	\end{figure}

\subsubsection{Parametre}
\label{sec:parametre}

\textbf{Parametre} sú~spôsob, ako môže komunikovať užívateľ so~základnými nastaveniami uzlov
bez~potreby zmenenia kódu a~jeho následnej kompilácie, čo pri~väčších projektoch môže zabrať
aj~trištvrte hodiny. Konfigurácie sa definujú v~\textit{yaml} konfiguračnom súbore. V~ňom si
môžeme zadefinovať mená jednotlivých parametrov a~ich základné hodnoty. Tie si programátor
vie v~programe vytiahnuť pomocou API, \acrlong{API}, (Aplikačné Programovacie Rozhranie) v~ROSe.

\subsection{ROS1}
\label{subsec:ros1}

ROS bol prvýkrát vydaný v~roku 2007. Jeho tvorcovia \textit{Keenan Wyrobek} and \textit{Eric Berger} zo~Stanford
univerzity zacali ROS ako osobny projekt za~ucelom odstranenia procesu znovuobjavovania kolesa~\cite{rosHistory}.
Ide o~softvér, ktorý sa začal vyvíjať so~zámerom zjednodušiť a~zjednotiť programovanie a~ovládanie robotov.
Od~doby, kedy vznikol prešiel mnohými verziami a~úpravami. Jeho neoddeliteľnou súčasťou sú~štrukturovanie programu
do~uzlov (nodov), komunikácia medzi uzlami, podpora viacerých programovacích jazykov ako sú~C, C++ alebo Python
a~vytváranie balíčkov dostupných širokej verejnosti.

Štrukturalizovanie základov ROS1 je spravené monoliticky. Tvorcovia pritom dbali na~to, aby tento
systém vytvorili čo najstabilnejším spôsobom. Na~počiatku musí byť spustený hlavný program (roscore),
ktorý zabezpečuje vytváranie jednotlivých uzlov. Komunikácia medzi uzlami je zabezpečená prostredníctvom
prepojenia uzlov cez~LAN/WLAN alebo IPC komunikáciu. Ak~sú~uzly spustené na~iných zariadeniach,
tak~sa využíva len komunikácia cez~sieť. Roscore ďalej poskytuje parametre jednotlivým uzlom
z~parametrového servera. Jeho najdôležitejšou úlohou je zabezpečenie komunikácie uzlov v~programe.

\clearpage

Aj~napriek mnohým výhodám má ROS1 nedostatky, ktoré sa ťahajú už~od~jeho počiatkov. Sú~to~napríklad:

\begin{itemize}
	\item Nepostačujúca distribuovanosť systému. Všetky uzly sa spoliehajú na~funkčnosť roscore-u,
	\item ROS1 je písaný v~starom štandarde, to~vnáša do~programu technologický dlh a~bezpečnostné riziká,
	\item Kvalita komunikácie sa nedá ovplyvniť,
	\item Preddefinované vláknové moduly~\cite{ROS2design},
	\item Možnosť užívateľa predefinovať základne prvky ROS-u.
\end{itemize}

Kvôli takýmto problémom a~nedostatkom sa začala vyvíjať nová verzia ROSu, ROS2. Tá mala vyriešiť tieto problémy a~zlepšiť funkcionalitu
prvej verzie. V~roku 2025 sa skončí podpora poslednej distribúcie ROS1 menom \textit{Noetic}. Preto~je odporúčané začínať nové projekty
v~ROS2.

\subsection{ROS2}
\label{subsec:ros2}

\begin{figure}[!htbp]
	\centering
	\includegraphics[width=15cm]{img/strukturaRos1Ros2.png}
	\caption{Porovnanie štruktúr ROS1 a~ROS2~\cite{comparison}}
	\label{fig:struktury}
\end{figure}

Ako už~bolo spomenuté zámerom vývoja ROS2 bolo zlepšenie funkcionality a~bezpečnosti systému. Jeden z~dôsledkov tohto vývoja je, že~ROS2
nie je spätne kompatibilný software. Podstata toho, ako sú~zoskupované uzly a~ako spolu komunikujú je diametrálne odlišná od~ROS1. Z~tohto dôvodu
bol vyvinutý takzvaný rosbridge, ktorý zabezpečuje kompatibilitu medzi verziami. Nie je to~ale~trvalé riešenie. Odporúčané je nástroj
využívať a~počas toho prepisovať kód z~verzie 1 do~verzie 2. Komunikácia prebieha v~ROS2 rovnakým spôsobom ako v~ROS1. Pomocou tém
\ref{sec:topic}, služieb \ref{sec:services} a~akcií \ref{sec:actions}.

Táto podobnosť končí na~najvyššej vrstve. Ako sme videli na~Obr.~\ref{fig:struktury}. Štruktúra ROS2 je rozdelená do~viacerých vrstiev.
Najdôležitejšie je pre~nás vedieť, že~komunikácia je spracovávaná modelom DDS (Služba distribúcie údajov) z~anglického
(\acrlong{DDS}). DDS obsahuje podobné dizajnové vzory ale~poskytuje väčšiu ovládateľnosť komunikácie systémov, zatiaľ
čo odstraňuje všetky vrstvy, ktoré zabraňujú debugovaniu~\cite{ElectronicDesign}. Tento model zároveň zlepšuje výkon,
stabilitu a~bezpečnosť modelu oproti ROS1. Je založený na~TCP/IP a~UDP/IP protokole. Z~obrázku Obr.~\ref{fig:struktury}
vyčítame aj~lepšie rozloženie modulov. To~zabezpečuje jednoduchšie prispôsobovanie systému pre~nové funkcionality. Podpora operačných systémov
sa v~ROS2 rozšírila aj~o~Windows, Mac OS či~RTOS (Operačné systémy reálneho času) z~anglického \acrlong{RTOS}. Operačné systémy nie sú~jediné
rozšírenie ohľadom kompatibility. S~ROS2 je možné programovať už~aj~v~Jave či~Matlabe. Tvorcovia mysleli aj~na~programátorov a~pridali rozšírené
možnosti testovania, debugovania či~nasadzovania programu do~reálneho využitia.

Testovanie prebieha pomocou použitia Google testov. Debugovanie je možné uskutočniť pomocou debuggera gnu-gdb. Pri~spustení programu
cez~spúšťací súbor (launch file) je potrebné pridať príkaz na~spustenie spomenutého debugovacieho programu.

ROS2 má necentralizovanú štruktúru, a~preto~pri~spúšťaní programov už~nie je potrebné mať spustený roscore. Ak~teda spadne jeden uzol všetky
ostatné uzly budú fungovať naďalej. V~ROS1 sme vedeli ovplyvniť počet uchovaných správ komunikácie pokým nepretiekol zásobník, ktorý ich uchovával
na~neskoršie použitie. V~ROS2 vieme implementovať túto schopnosť použitím takzvanej \textit{QoS} triedy (kvalita komunikácie), z~anglického \acrlong{QoS}.
Pomocou tejto triedy vieme zmeniť kvalitu komunikácie. Vieme si zadefinovať, či~by sme radšej stratili niektoré správy, ale~dostali by~sme
všetky rýchlo. Alebo aby sa zabezpečilo, že~dostaneme všetky správy, ktoré boli vyslané, aj~keby to~trvalo dlhšie. Dokonca si vieme zadefinovať
maximálny čas, ktorý budeme čakať na~ďalšiu správu.

Ak~by bol užívateľ veľmi schopný programátor a~potreboval by si zmeniť triedy, ktoré definujú základnú funkcionalitu ROS-u, tak~aj~toto je možné.
Jednou z~takýchto funkcionalít je, že~užívateľ si vie predefinovať triedu, ktorá bude alokovať miesto na~(IPC) komunikáciu (medziprocesovú komunikáciu).
K~tomuto bodu je dodať, že~tento prípad je špecifický a~väčšina programátorov sa s~takouto možnosťou do~kontaktu nedostane.

Pri~všetkých týchto zlepšeniach nemôžeme zabudnúť spomenúť aj~nasledovný nedostatok. Keďže ROS2 je mladší ako ROS1 nájdeme k~nemu menej dokumentácie.
Pridaním veľkého počtu funkcionalít začal vznikať problém pre~začiatočníkov s~porozumením niektorých kódov. Avšak tento problém je nedostatkom,
ktorý časom zanikne. V~čase písania tejto práce pribudli na~stránke dokumentácie minimálne 2 strany popisujúce pokročilejšie Funkcionality druhej
verzie ROSu.

\subsection{Rozdiely}

Čo je určite dobrou správou pre~všetkých programátorov, ktorí robili v~prvej verzii a~sú~zvyknutí na~jej štandardy a~funkcionalitu. Tak~títo
sa nemajú čoho obávať. Prechod z~ROS1 na~ROS2 je dosť priamočiary. Čo sa zmenilo je spôsob písania kódu, ale~koncepty ostali všetky rovnaké.
V~tejto sekcii nebudeme písať konkrétne kódy, budeme len opisovať čo je podobné a~čo zasa rozdielne medzi verziami spomínaného systému. Keďže
celý projekt bol písaný v~programovacom jazyku C++ tak~sa aj~tieto zmeny budu týkať hlavne jazyka C++.

\subsubsection{Štandard jazyka}

	Pokým ROS1 bola písaná v~štandarde C++03 tak~ROS2 je už~písaná v~novom štandarde. A~to~hlavne C++11, ale~používa aj~nejaké časti z~C++14
	a~C++17. To~zahŕňa inicializovanie templatov a~ich používanie. Tým, že~ROS2 je stále nová a~stále vyvíjajúca sa platforma, tak~môžeme očakávať aj~časti
	kódu, ktoré budú podporovať najnovší C++ štandard a~to~štandard z~rokov 2020 a~2023.

	Definície a~deklarácie templatov sú~na~knihu samú o~sebe, preto~do~detailov nebudeme zachádzať. Stačí nám vedieť, ako ich inicializovať.
	V~prvej verzii sme definovali všeobecného publishera (publikovateľa) a~definovali sme mu len cez~akú tému má posielať správy. V~druhej verzii
	naväzujeme publishera na~špecifický tip správy akú posielame. Nemôže sa teda stať, že~takýto program by sme skompilovali a~následne, keď
	ho spustíme, tak~by spadol z~dôvodu, že~čítame iný typ správy ako posielame.

\subsubsection{Inicializácia nody (uzla)}

	Tak~isto ako v~prvej verzii aj~v~druhej verzii musíme definovať uzol (node). Rozdiel je v~tom, že~prvá verzia obsahovala NodeHandle (Ovládač uzla)
	a~druhá verzia obsahuje priamo Node (Uzol). V~druhej verzii je zaužívaným štandardom túto nodu predediť a~použiť polymorfizmus pri~objekte,
	ktorý bude existovať počas celej doby vykonávania programu. Pri~prvej verzii tomu tak~nebolo. Tam sme museli vytvoriť už~spomenutý NodeHandle.
	Ten sa nemusel využiť ako base trieda a~nemusel ani~existovať počas celého behu programu.

\subsubsection{Komunikácia}

	DDS (Služba distribúcie údajov) je protokol strednej vrstvy (middleware) implementovaný nad
	UDP~\cite{ROS2book}.Tento protokol je použitý v~IoT (internet veci) (\acrlong{IoT}) sfére.
	Využíva sa napríklad v~oblastiach ako sú~telekomunikácie, vizualizácia, automatizácia, alebo
	zdravotníctvo~\cite{siteTrend}. DDS je používaný v~ROS2 na~komunikáciu medzi uzlami.
	Je to~systém správ publikovania (publish) / odoberania (subscribe), ktorý umožňuje uzlom
	komunikovať medzi sebou bez toho, aby poznali identitu ostatných uzlov. Druh komunikácie
	je v~ROS2 rozšírený ešte o~akcie viď kapitolu~\ref{sec:actions}.

\subsubsection{Parametre}

	ROS1 používa parametrový server, ktorý sa nachádza v~roscore-e. Každý uzol si mohol vytiahnuť parametre, ktoré boli zapísané v~konfiguračnom
	súbore. ROS2 žiadny roscore nemá, preto~sa parametre musia distribuovať iným spôsobom. Parametre v~druhej verzii ROSu patria jednotlivým uzlom.
	To~znamená, že~jednotlivé parametre sa dajú vytiahnuť len daným uzlom. Tieto parametre taktiež existujú len počas existencie daného uzlu. Parametre
	sú~ďalej distribuované pomocou už~spomínaného DDS protokolu. V~prípade, že~sa tieto parametre nepodari vytiahnuť z~konfiguračného súboru. Či už~z~dôvodu,
	že~daný súbor neexistuje, alebo iného dôvodu, tak~sa aplikujú základné hodnoty, ktoré si zvolil užívateľ pri~používaní funkcie na~ich zisťovanie.

\subsubsection{Nodelet alebo komponent}

	ROS1 ponúka možnosť definície uzlov ako uzlík (\texttt{nodelet}). Je to~definovanie uzlu ako zdielanej knižnice (shared library). Je
	to~spôsob ako uľahčiť prácu CPU. Keď sa definuje uzol ako uzlík, tak~jeden proces môže spracovávať programy z~viacerých takýchto uzlíkov.
	Táto funkcionalita sa nachádza aj~v~ROS2. Volá sa komponent (\texttt{component}). Vylepšením oproti nodelet-om je zjednotenie aplikačnej
	implementácie (API). Pokým nodelet-y mali vlastný spôsob implementácie v~ROS1 tak~v~ROS2 je implementácia uzla a~komponentu rovnaká.
	Pri~komponente sa musí len naviac definovať, že~daný komponent existuje pomocou makra. Použitie komponentov zjednodušuje prácu CPU
	a~používa sa hlavne v~zariadeniach, ktoré majú obmedzený výkon výpočtovej techniky. Sú~to~napríklad mikroprocesory, ktoré ovládajú roboty.

\subsubsection{Kompilácia}

	Zmenou verzii sa zmenil aj~spôsob kompilácie programu. ROS1 bol kompilovaný pomocou \texttt{catkin} build systému. Catkin je založený
	na~programe \texttt{cmake}. Jeho nastavenie dependencií je konfigurované pomocou súboru \texttt{package.xml}. ROS2 prešiel na~viac
	nastaviteľný systém \texttt{Colcon}. Tento systém je na~rozdiel od~catkin-u založený na~Python-e a~jeho dependencie sa nastavujú pomocou
	\texttt{setup.py} súboru. V~prípade colcon-u si môžeme definovať spôsob kompilácie to~znamená, že~môžeme nastaviť, ako sa budú spracovávať
	dependencie. Ponúkané možnosti sú~\texttt{catkin\_make}, \texttt{catkin\_make\_isolated}, \texttt{catkin\_tools} a~\texttt{ament\_cmake}.
	Jednou s~najviac používaných možností je \texttt{ament\_cmake}. Je založený na~programe \texttt{cmake} a~spolupracuje so~systémom \texttt{colcon}.
	Z~tohto dôvodu mu vieme definovať dependencie pomocou xml súboru ako tomu bolo v~ROS1 pričom možnosť definície pomocou Python skriptu ostáva.
	Je to~jeden zo~spôsobov, ako zmenšiť rozdiel medzi ROS1 a~ROS2.

\subsubsection{Vlákna}

	ROS1 dovoľuje programátorom vybrať si medzi jedno vláknovým a~viac vláknovým vykonávaním programu. Tvorcovia ROS2 si dali zaležať na~modularite
	aj~tejto oblasti kódu. V~druhej verzii ROS-u si vieme zadefinovať typ vykonávania programu separátne pre~každý uzol a~vieme si tento typ
	zadefinovať aj~sami~\cite{ROS2design}.

\subsection{Ros control}
\label{subsec:roscontrol}

Roc\_control je balík naprogramovaný v~prostredí ROS1 aj~ROS2. Tým, že~sme si vybrali ROS2 ako prostredie pre~naše riešenie,
tak~by sme použili balík ros2\_control. Obrazok \ref{fig:components_architecture} zobrazuje architektúru balíka ros2\_control.

\begin{figure}[!htbp]
	\begin{center}
		\includegraphics[width=0.95\textwidth]{./img/components_architecture.png}
	\end{center}
	\caption{Diagram zobrazujúci vnútornú implementáciu balíka ros2\_control~\cite{roscontrol}.}
	\label{fig:components_architecture}
\end{figure}

Delegátor snímačov \textbf{Controller Manager} (CM) prepája snímače a~abstrakciu hardvéru balíka ros2\_control. Slúži
taktiež ako vstupný bod pre~užívateľov ROS služieb~\cite{roscontrol}.
Delegátor snímačov na~jednej strane ovláda snímače a~rozhrania, ktoré potrebujú. Na~druhej strane má pristúp k~hardvérovým
komponentom, t.j. ich rozhraniam. Delegátor snímačov spája požadované a~poskytované rozhrania, poskytuje pristúp ovládačom
k~hardvéru, keď ho potrebujú alebo hlási chybu, ak~existuje konflikt prístupu~\cite{roscontrol}.

Delegátor zdrojov \textbf{Resource Manager} (RM) spravuje fyzický hardvér a~ich ovládače (hardvérové komponenty)
pre~balík ros2\_control~\cite{roscontrol}. V~tejto triede je možnosť definovať takzvané \textit{callbacky}
(ukazovatele na~funkcie, ktoré môže užívateľ definovať) \textit{read()} a~\textit{write()}, ktoré zabezpečujú
komunikáciu hardvérových komponentov~\cite{roscontrol}.

Hardvérové komponenty \textbf{hardware components} (HC) sú~triedy, ktoré realizujú komunikáciu s~fyzickým hardvérom
a~jeho reprezentáciou v~balíku ros2\_control~\cite{roscontrol}.

\clearpage

V~tomto balíku sú~definované tri základné typy komponentov:

\begin{itemize}
	\item \textbf{Systém} Komplexný (multi-DOF (\acrlong{DOF}) (z~anglického veľa stupňov voľnosti)) robotický hardvér ako industriálny
		robot. Hlavný rozdiel medzi akčným členom a~možnosťou použitia komplexného prechodu aký potrebujeme pri
		Robotickým ramene ruky. Komponent poskytuje čítanie a~zápis do~hardvéru. Používa sa v~len prípade jednej
		logickej komunikácie kanálov hardvéru (napríklad KUKA-RSI)~\cite{roscontrol}.
	\item \textbf{Senzor} Robotický hardvér prizvaný na~zaznamenanie okolitého prostredia. Senzor ako komponent je
		príbuzný s~kĺbom (napríklad prekladač) alebo rameno (napríklad snímač sily a~krútiaceho momentu). Tento
		komponent ma len schopnosť čítania vlastnosti~\cite{roscontrol}.
	\item \textbf{Akčný člen} Jednoduchý (1 DOF) robotický hardvér ako motor, chlopňa alebo podobný člen.
		Akčný člen implementuje len jeden kĺb. Tento komponent má schopnosť čítať aj~zapisovať vlastnosti. Čítanie nie
		je požadované ak~nie je uskutočniteľné (napríklad DC ovládač motora s~Arduino doskou). Tento akčný člen môže byť
		tiež použitý s~viac stupňovými kĺbmi voľnosti robota ak~jeho hardvér umožňuje tento dizajn (napríklad
		CAN-komunikácia s~každým motorom nezávisle)~\cite{roscontrol}.
\end{itemize}

Aj~napriek všetkým týmto výhodám nemôžeme tento softvér použiť. Je to~kvôli tomu, že~na~robote sa systém ROS nenachádza
a~nemá preto~pristúp k~jednotlivým častiam hardvéru. Ak~by tento systém na~robote bol implementovaný mohli by sme použiť
funkciu \textbf{Delegátora zdrojov} na~cicanie a~zapisovanie dát do~jednotlivých komponentov. Mrime si preto~vytvoriť
vlastný systém na~čítanie, spracovanie a~ovládanie robota.



\section{Pôvodný stav robota}

\subsection{Robot a~jeho ovládanie}

Robot, s~ktorým sme pracovali bol výsledkom tímového projektu viacerých študentov \newline z~roku 2019. Pri~vysvetľovaní a~opisovaní robota sa budeme
odvolávať na~dokumenty, stránky a~kód, ktorý napísali. Všetky tieto údaje si sprístupnené na~mobilnom robote v~záložke
\newline \texttt{\$(HOME)/Desktop/Blackmetal}~\cite{timovyProjekt}.

Robot je v~tvare kvádra. Jeho šírka je 60cm a~je vyzdvihnutý nad zem o~1.5cm. Nachádza sa na~kolesách o~polomere 8cm. Jeho kostra, až na~oceľové pláty,
ktoré držia robot, je spravená z~hliníku. Konkrétne z~hliníkových tyčí, ktoré sú pospájané plexisklovými plátmi. Jeho podobizeň vidíme na~nasledujúcom
obrázku.

\begin{figure}[!htbp]
	\begin{center}
		\includegraphics[width=0.95\textwidth]{img/robot.png}
	\end{center}
	\caption{Zobrazenie spodnej časti mobilného robota~\cite{timovyProjekt}}
	\label{fig:robot}
\end{figure}

\noindent Na~obrázku ďalej vidíme olemovanie robota pásom s~LED-kami. Tie svietia nasledovným spôsobom. Keď sa robot nehýbe všetky LED-ky svietia
na zeleno. Keď sa robot pohne do~nejakej strany, LED-ky znázornia jeho pohyb tým, že~svietia na~strane, do~ktorej sa robot hýbe. Keď nastane
situácia, kedy počítač ovládajúci motory prestane komunikovať s~Arduinom, ktoré sa stará o~detekciu stavov robota tak~LED-ky začnú blikať
červeno-modrými farbami.

Ako bolo spomenuté LED-ky znázorňujú pohyb robota. Ten sa pohybuje za pomoci diferenciálneho podvozku s~dvoma podpornými všesmerovými kolesami.
Motory robota sú pripojené na~meniče. Tie sú ovládané priamo príkazmi z~počítača.

Hardware robota sa skladá z:
\begin{itemize}
	\item kontrolnej dosky Arduino Uno,

	\item Počítača ADVANTECH MIO-5272~\cite{robotPc} \newline
		Počítač obsahuje operačný systém Ubuntu 16.04.

	\item Extension board MIOe-210~\cite{extensionModule}

	\item Meniče MAXON EPOS 24/5 (s číslom 275512)~\cite{menic} \newline
	 	Sú napájané jednosmerným napätím 11 - 24 V~a~5 A.

	\item Enkódery MAXON Encoder MR Type L (s číslom 225787)~\cite{encoder} \newline
		Rozlíšenie enkóderov je 1024 impulzov s~troma kanálmi.

	\item Motory MAXON RE 40 (s číslom 148867)~\cite{motor} \newline
		Motory s~výkonom 150W. Maximálna rýchlosť je 12 000 rpm a~efektivita 91\%.

	\item Prevodovka MAXON Planetary Gearhead GP 42 C (s číslom 202120)~\cite{prevodovka} \newline
		Redukcia prevodovky je 43:1. Jej účinnosť je 72\%.
\end{itemize}

\noindent Ovládanie robota je zabezpečené externými počítačmi
\begin{itemize}
	\item Control PC (Kontrolný počítač) -- Počítač posielajúci príkazy na~robot cez~TCP/IP protokol.
	\item Logging PC (Logovací počítač) -- Počítač prijímajúci stav robota cez~TCP/IP protokol.
\end{itemize}

\begin{figure}[!htbp]
	\begin{center}
		\includegraphics[width=9cm]{img/schemaRobota.png}
	\end{center}
	\caption{Schéma zapojenia jednotlivých častí na~robote}
	\label{fig:schemaRobota}
\end{figure}

\noindent Tieto počítače sú len reprezentácia servera. V~realite to môže byť jeden a~ten istý počítač.

\noindent Na obrázku Obr.~\ref{fig:schemaRobota} vidíme zapojenie jednotlivých častí robota. Čo sme nespomenuli a~je na~obrázku je XBox ovládač
je to~kvôli tomu, že~tímový projekt bol zameraný na~ovládanie robota pomocou tohto ovládača. My ho ale použivať nebudeme.

\subsection{Komunikácia s~robotom}

S robotom sa vieme spojiť pomocou dvoch portov. Jeden port je otvorený na~prijímanie požiadavok (requestov) a~ten druhý je na~monitorovanie
stavu robota. Port \textit{664} je otvorený pre~tisíc užívateľov, ktorí môžu len sledovať stav robota. Druhý port je na~prijímanie requestov
\textit{665} a~je otvorený len pre~jedného užívateľa.

\subsubsection{Logovanie}
\label{sec:logovanie}

	Spomínaný port \textit{664} je otvorený jednému užívateľovi. Keď sa užívateľ pripojí začne dostávať nepretržite správy typu
	\textit{JSON}~(\acrlong{JSON}), ktoré hlásia stav robota. Správy, ktoré dostávame sú nasledujúceho formátu

	\begin{lstlisting}
			{"state":1,"direction":1}
	\end{lstlisting}

	Hodnoty sa pri~stave (state) a~ani pri~smere (direction) nemenia. Sú to~stále jednotky. Pokým robota nezastavíme buď príkazom, stlačením
	tlačidla vypnutia alebo zablokovaním jedného z~kolies, tak~sa tieto správy budú posielať. Môžeme potom začať polemizovať o~tom či~by nebolo
	lepšie už tieto správy využiť na~to~čo reálne spomenutý \textit{JSON} reťazec ukazuje. A~to udávať smer a~stav robota. Momentálne tieto správy
	slučia len na~to, aby sme vedeli, že~tento robot je aktívny a~vie primať a~spracúvať informácie.

\subsubsection{Ovládanie}
\label{sec:ovladanie}

	Port \textit{665} je sprístupnený na~prijímanie a~odosielanie požiadavok a~ich odpovedí. Príkazy sa na~počítač posielajú cez~sieť z~externého
	počítača vo~formáte \textbf{JSON}. Študenti, ktorí navrhovali systém posielania požiadavok (request) a~odpovedí (response) robili tieto správy
	ručne. Preto~nastávajú situácie, kedy robot pošle správu, ktorá nespadá do~štandardu písania JSON textu. Z~tohto dôvodu sme nemohli použiť už
	existujúci kód (parser), ktorý by nám zjednodušil prehľadávanie týchto správ. Podla dokumentácie sa robot mal ovládať správami typu~\cite{BMdoc}

	\label{jsonSpeedRequestBad}
	\begin{lstlisting}
			{"UserID":1,"Command":3,"RightWheelSpeed":50,"LeftWheelSpeed":50}
	\end{lstlisting}

	\newpage

	\noindent Význam jednotlivých parametrov:
	\begin{itemize}
		\item \textbf{UserID} -- Znázorňuje ID užívateľa, ktorý je pripojený na~robot. Predvolená hodnota je 1.
		\item \textbf{Command} --  Číselná hodnota znázorňujúca príkaz, ktorý ma robot vykonať
			\begin{enumerate}
				\setcounter{enumi}{-1}
				\item \label{c0} Prázdny príkaz slúžiaci na~overenie spojenia
				\item \label{c1} Núdzové zastavenie
				\item \label{c2} Normálne zastavenie
				\item \label{c3} Príkaz nastavujúc rýchlosti kolies mobilného robota
				\item \label{c4} Prázdny príkaz
				\item \label{c5} Prázdny príkaz
				\item \label{c6} Príkaz pýtajúci si aktuálnu rýchlosti pravého a~ľavého kolesa. Tento príkaz nebol sprave navrhnutý v~kóde
					robota. Vracal nám žiadaná hodnotu namiesto aktuálnej. Museli sme ho prepísať.
				\item \label{c7} Pripravenie motorov robota
				\item \label{c8} Príkaz pýtajúci si aktuálnu pozíciu pravého a~ľavého kolesa.
			\end{enumerate}
		\item \textbf{RightWheelSpeed} -- Nastavenie rýchlosti pre~pravé koleso
		\item \textbf{LeftWheelSpeed} -- Nastavenie rýchlosti pre~ľavé koleso
	\end{itemize}

	\noindent Z~tohto kusu kódu je jasné, že~sa majú posielať celé čísla a~na základe tohto vstupu sa bude robot hýbať. Čo sme zistili až
	po skompilovaní a~spustení tímového projektu je, že~sa majú posielať desatinné čísla z~intervalu 0 až 1. Toto nebolo písané
	v~dokumentácii, ktorá nám bola dodaná na~začiatku programu. Môžeme preto príklad prepísať na~reťazec, ktorý by fungoval

	\label{jsonSpeedRequestGood}
	\begin{lstlisting}
			{"UserID":1,"Command":3,"RightWheelSpeed":0.50,"LeftWheelSpeed":0.50}
	\end{lstlisting}

\subsection{Vysvetlenie kľúčov reťazca}

\noindent \textbf{UserID} \newline
\indent Táto možnosť je v~momentálnom stave robota nevyužitá. Počet zariadení, ktoré sa môžu pripojiť na~port, cez~ktorý sa dá robot ovládať
je 1. Je to~ale dobrá možnosť na~rozšírenie kódu. Keď sa budú môcť pripojiť viacerí užívatelia, tak~sa bude musieť vyriešiť, koho príkaz
bude mať akú prioritu. \newline

\noindent \textbf{Command:~\ref{c4}} \newline
\indent Tento príkaz je prázdny. My sme ho ale neskôr prepísali na~príkaz, cez~ktorý sa dá nastaviť žiadaná pozícia kolies robota (natočenie).
Táto funkcionalita nie je v~takom stave ako sme si priali. Je to spôsobené hlavne nedostačujúcov dokumentáciou enkóderov na~robote. Síce sme našli
v~dokumentácii funkciu, ktorá by mala túto možnosť povoľovať. Čo sa ale stane pri~poslaní príkazu je to, že~kolesá sa začnú točiť rýchlosťou
0,5 metra za sekundu.\newline

\noindent \textbf{Command:~\ref{c8}} \newline
\indent Príkaz na~zisťovanie polohy kolies nebol originálne naprogramovaný na~robote. Pridali sme ho za cieľom presného dostavia sa robota
na~preddefinované miesto. Táto funkcionalita nefunguje správne rovnako ako v~predchádzajúcom príklade, keď si vypýtame polohu kolies od robota,
dáta ktoré obdržme sú, že~jedno koleso je priamo nastavené na~hodnotu, ktorú sme si vyžiadali a~to druhé koleso vráti náhodnú hodnotu.
Počas toho sa ale kolesá robota stále točia.\newline

\noindent \textbf{RightWheelSpeed/LeftWheelSpeed} \newline
\indent Nastavovanie rýchlosti pravého a~ľavého kolesa nie sú povinné parametre. Musíme ich zadávať len v~prípade posielania rýchlostí
cez~príkaz s~číslom \ref{c3} alebo \ref{c4}.

\subsection{Oprava chýb na~robote}

\subsubsection{Nesprávna funkcia}

Ako bolo spomenuté vyššie, pri~poslaní príkazu s~číslom~\ref{c6} nám robot vráti aktuálne rýchlosti kolies. Počas skúšania tejto funkcionality
sme narazili na~problém. Keď sme sa robota spýtali na~jeho rýchlosti. Dostali sme reťazec, ktorý obsahoval náhodne veľké čísla. Tieto čísla sa
menili, keď sme zadávali nejaké hodnoty pre~rýchlosti kolies aby~sa robot hýbal. Ich magnitúda ostávala rovnaká. V~nasledujúcom príklade môžeme
vidieť ako tento reťazec vyzeral:

\label{jsonWannabeSpeed}
\begin{lstlisting}
		{"LeftWheelSpeed"=236223201280 "RightWheelSpeed"=4294967296}
\end{lstlisting}

Tu vidíme príklad obdržanej správy. Ako si môžeme všimnúť. Pri~tomto type správ nie je dodržaná správna forma reťazca typu JSON. Namiesto `:'
máme `=' a~medzi argumentmi sa nenachádza čiarka. Hneď ako prvú vec sme chceli tento štandard napraviť. Bohužiaľ na~tomto robote už~bolo
spravených niekoľko projektov a~museli by sme prejsť každý z~nich a~zistiť či~používajú túto spätnú väzbu. Ak~by ju používali museli by sme
tieto kódy upraviť.

\newpage

V~dokumentácii robota bohužiaľ nebolo písané v~akom formáte sa tieto rýchlosti kolies majú nachádzať. Preto jeden z~nápadov ako zistiť presne
v~akom formáte sa posielali tieto čísla bolo vyskúšať pár možností. Boli to

\begin{itemize}
	\item \textit{long} - celé číslo s malým endianom
	\item \textit{long} - celé číslo s veľkým endianom
	\item \textit{float} - desatinné číslo s malým endianom
	\item \textit{float} - desatinné číslo s veľkým endianom
\end{itemize}

Keďže robot má počítač so~63 bytovým procesorom \cite{robotPc}, tak~\textit{long} aj~\textit{float} budú mať 64 bitovú dĺžku. Po~skúsení všetkých
štyroch možností sa~ukázalo, že~ani jedna nebola správna a~problém je niekde inde.

Problém je v~tom, že~keď posielame request na~nastavenie rýchlosti kolies, tak~kód na~robote funguje tak, že~si ich premení na~celé čísla v~rozsahu
0~až~1000. To~je hodnota, na~ktorú nastaví rýchlosti otáčania pravého a~ľavého kolesa respektíve rýchlosť otáčania ich motorov. Na~druhú stranu,
keď si vypýtame od~robota rýchlosti kolies. On zoberie informáciu z~enkóderov a~pošle nám ju~bez spracovania. Aj napriek týmto poznatkom sa nám
nepodarilo získať z~týchto dát žiadané rýchlosti.

Po dôkladnom preštudovaní kódu sme zistili, že~hodnoty ktoré nám posiela robot nie sú ani~vyťahované z~enkóderov správnou funkciou. Preto~sme ju
zmenili a~začali sme dostávať hodnoty, s~ktorými by sa mohlo dať pracovať.

Funkcie z~knižnice zabezpečujúce komunikáciu z~enkóderov motorov pochádzajú z~firmy Maxon~\cite{EPOSdoc}. Táto dokumentácia nebola moc nápomocná.
Opisy jednotlivých funkcií boli len~ich rozložené názvy na~osobitné slová. Aj~napriek tomu sa nám podarilo nájsť funkcie, ktoré sme potrebovali.
Funkcie, ktoré končia koncovkou `Target', alebo toto slovo obsahujú, majú návratné hodnoty reprezentujúce žiadané hodnoty. Funkcie s~koncovkou
`Is' vracajú aktuálne hodnoty. Z~tohto dôvodu sme museli prepísať funkciu na~robote, ktorá sa vykonávala, keď sme chceli získať aktuálne hodnoty
rýchlosti motora poslaním príkazu \ref{c6}. Funkciu, ktorú sme zmenili môžeme vidieť v~nasledujúcej ukážke:

\lstset{language=C++,
	basicstyle=\ttfamily,
	keywordstyle=\color{blue}\ttfamily,
	stringstyle=\color{red}\ttfamily,
	commentstyle=\color{green}\ttfamily,
	morecomment=[l][\color{magenta}]{\#},
	numberstyle=\color{orange}
}

\label{VelocityIs}
\begin{lstlisting}[language=C++]
BOOL VCS_GetTargetVelocity(
	HANDLE KeyHandle,
	WORD NodeId,
	long* pTargetVelocity,
	DWORD* pErrorCode);
\end{lstlisting}

\begin{lstlisting}[language=C++]
BOOL VCS_GetVelocityIs(
	HANDLE KeyHandle,
	WORD NodeId,
	long* pVelocityIs,
	DWORD* pErrorCode);
\end{lstlisting}

\noindent Ako môžeme vidieť v~týchto predpisoch funkcií, bolo treba zmeniť názov funkcie a~ostatné parametre ostali rovnaké.
Nebolo treba meniť implementáciu kódu.

\subsubsection{Zašumený výstup}

Po~prepísaní funkcie na~získavanie rýchlostí robota sme spravili pár meraní, aby sme zistili, aké presné informácie o~rýchlostiach
motorov dostávame. Aby nám robot neodbiehal postavili sme ho na~vyvýšené miesto, tak~aby sa kolesá nedotýkali zeme. V~takomto
postavení sa robot nepohne z~miesta a~my môžeme bez~problémov odmerať prechodové a~prenosové charakteristiky rýchlosti pravého
a~ľavého motora.

\begin{figure}[!htbp]
	\begin{subfigure}{0.5\textwidth}
		\includegraphics[width=\textwidth]{img/Left_wheel_2.png}
	\end{subfigure}
	\hfill
	\begin{subfigure}{0.5\textwidth}
		\includegraphics[width=\textwidth]{img/Right_wheel_2.png}
	\end{subfigure}
	\caption{Ustálené hodnoty rýchlosti ľavého a~pravého motora. }
	\label{fig:lavePraveKoleso}
\end{figure}

Po~obdržaní takýchto dát sme kontaktovali jedného z~autorov tímového projektu~\cite{timovyProjekt}, Adriána Kasperkevic. On nám odpísal s~tým,
že~aj oni mali problémy s~enkódermi. Na obrázku Obr.~\ref{fig:prechChar} vidíme zašmený signál rýchlosti poskytovanú enkódermi. Ich problémy
boli naviazané na~staré enkódery, ktoré neskôr vymenili. Na~nových enkodéroch avšak netestovali ich spätnú väzbu.

\begin{figure}[!htbp]
	\begin{center}
		\includegraphics[width=0.95\textwidth]{img/robotSpeedChar.png}
	\end{center}
	\caption{Prechodová charakteristika rýchlosti kolies~\cite{timovyProjekt}. }
	\label{fig:prechChar}
\end{figure}

\section{Implementácia ovládača}
\label{sec:program}

\begin{figure}[!htbp]
	\begin{center}
		\includegraphics[width=0.9\textwidth]{img/BlackMetal_flowchart.png}
	\end{center}
	\caption{Graf vykonávania programu na~ovládanie robota pomocou ROS2.}
	\label{fig:flowchart}
\end{figure}

\subsection{Úvod do~čítania grafu}
\label{sec:citanie_grafu}

Na~obrázku Obr.~\ref{fig:flowchart} môžeme vidieť viacero objektov rôznych farieb. Objekty zobrazené červenou farbou sú uzly spracovávané
a~vytvárané v~rámci ROS2. Každý tento objekt sa vykonáva v~osobitnom procese. Objekty zobrazené oranžovou farbou sú objekty, ktoré majú svoje
vlastné vlákno. Tieto objekty boli vytvorené uzlom BlackMetal. Dátové štruktúry Queue, zobrazené bielou farbou sú vytvorene  tak~aby
zabezpečovali bezchybnú komunikáciu medzi viacerými vláknami. Objekt so zelenou farbou je vstup do~programu. Je zadávaný užívateľom
a~reprezentuje žiadanú rýchlosť robota. Modry objekt je výstupom programu. Je to téma, na~ktorý sa publikuje aktuálna pozícia robota.
Robot samotný je zobrazený bielou farbou vo~forme malého obláčika. Prerušované čiary na~diagrame znázorňujú sieťovú komunikáciu programu
a~robota. Červené dvojité čiary udávajú, ktoré objekty patria ROS-u. Nakoniec čierne plné čiary reprezentujú tok dát medzi objektami.

\subsection{Uzly}

Na~obrázku Obr.~\ref{fig:flowchart} môžeme vidieť postup vykonávania programu na~ovládanie robota BlackMetal pomocou ROS2.
Na~začiatku programu sa vytvoria 3 uzly. Prvý uzol \textbf{Position Publisher Node} je uzol, na~ktorý sa publikuje vypočítaná pozícia robota.
Počíta sa na~základe obdržaných dát z~enkóderov robota. Uzol \textbf{Logger Node} slúži na~zaznamenávanie stavu robota \ref{sec:logovanie}.
Posledný uzol \textbf{BlackMetal Node} ovláda robota na~základe zadaných dát užívateľom.

\subsection{Vstup}

Uzol \textit{BlackMetal Node} vytvorí príjemcu, ktorý počúva na~téme \textit{geometry\_msgs/msg/Twist}. Tento vstup je vo~forme príkazu
zadaného v~príkazovom riadku. Vyzerá nasledovne:

\lstset{language=bash,
	basicstyle=\ttfamily,
	keywordstyle=\color{blue}\ttfamily,
	stringstyle=\color{orange}\ttfamily,
	commentstyle=\color{green}\ttfamily,
	morecomment=[l][\color{magenta}]{\#},
	numberstyle=\color{red}
}

\label{requestCommand}
\begin{lstlisting}[language=bash]
	ros2 topic pub /cmd_vel geometry_msgs/msg/Twist
		"linear:
			x: 0.0,
			y: 0.0,
			z: 0.0,
		angular:
			x: 0.0,
			y: 0.0,
			z: 0.0" -1
\end{lstlisting}

Tento \hyperref[requestCommand]{príkaz} publikuje jednu správu (\textit{-1}) o lineárnych a~uhlových rýchlostiach na~tému \textit{/cmd\_vel}
Táto sprava je typu \textit{geometry\_msgs/mgs/Twist}. Je následne spracovaná a~uložená do~rady \textit{Queue}. Tato rada je prioritne založená.
To znamená, že~požiadavka s nižším kódom ma vyššiu prioritu. Požiadavky a~ich kódy moceme vidieť v~sekcii \ref{sec:ovladanie}.

\subsection{Komunikacia s robotom}
\label{sec:robotComms}

Ako je naznačené na~Obr.~\ref{fig:flowchart}, klient si vo~svojom vlastnom vlákne vytiahne prvú spravu z~rady a~pretransformuje ju do~formy JSON.
Tento typ spravy môžeme vidieť v~\ref{sec:ovladanie}. Príkaz je poslaný robotu a~ten obratom dá vedieť, či danú požiadavku obdržal. Ak tato správa
žiadala rýchlosti kolies, tak~robot ďalšou správou odpovie na~danú požiadavku. Typ tejto odpovede môžeme vidieť v~\ref{jsonWannabeSpeed}. V~tomto prípade
sa správa spracuje a~uloží sa do~ďalšej rady.

\subsection{Odometria}
\label{sec:odometria}

Odometria, počítanie polohy na~základe rýchlosti kolies, sa vykonáva rovnako ako komunikácia s robotom v~separátnom vlákne. Tu je potreba si uvedomiť
jednu skutočnosť. To je tá, že~keď posielame žiadosť na~nastavenie rýchlostí kolies robota, tak~robot si hodnoty v~žiadosti prepočíta a~dáta dá následne
enkóderov. Keď si ale tieto rýchlosti vyžiadame z~enkóderov, tak~sa robot týchto dát nechytá a~my si ich musíme prepočíta na~metre za sekundu. Zároveň
si tento objekt neustále pýta od robota rýchlosti kolies. Ako sme už spomenuli v~\ref{sec:robotComms}, tieto spravy majú nižšiu prioritu ako nastavenie
rýchlosti kolies alebo bezpečnostné zastavenie robota. Preto sa môže stat, že~správy posielané robotu nebudú dodržiavať presne stanovenú frekvenciu v~čase
keď mu bude užívateľ posielať príkazy.

\subsection{Zdieľanie polohy}
\label{sec:zdielanie_polohy}

Ďalšiu vec, ktorú je treba vysvetliť ku grafu Obr.~\ref{fig:flowchart} je zdieľanie polohy. Odometria po~každom prepočítaní polohy robota publikuje tuto
informáciu na~tému \textit{/position} tato sprava je typu \textit{geometry\_msgs/msg/Vector3}. Obsahuje 3 hodnoty, ktoré sú x, y a~z.
Súradnice reprezentujú aktuálnu polohu robota.

\section{Filtrovanie zašumeného signálu}
\label{sec:ziskavanieRychlosti}

Ako bolo spomenuté v~predchádzajúcej kapitole, rýchlosti kolies sa dajú získať z~enkóderov. Tieto dáta sa posielajú v~správe, ktorá pripomína JSON formát.
Z~týchto vzoriek poslaných robotom nevieme priamo vypočítať polohu. Musíme si tieto dáta premeniť z~impulzov za sekundu \(\frac{1}{s}\) na~metre
za sekundu \(\frac{m}{s}\). Tento prevod nebude jednoznačný, pretože každý enkódery posiela dáta inak zašumené. Preto je potrebné zistiť, ako sa zmení
rýchlosť pri~zmene impulzov za sekundu. Tento prevod je možné získať z~merania, kde po~nastavení rýchlostí zoberieme veľa dát z~enkóderov a~zistíme,
ako sa zmení rýchlosť pri~zmene impulzov za sekundu.

Prvý nápad na~získanie čo najlepšej prevodovej charakteristiky bolo cez~všetky dáta položiť lineárnu regresiu. To sa ukázalo ako zlé riešenie, lebo dáta,
ktoré dostávame majú veľmi veľký rozptyl. Jednou z~nasledujúcich úvah bolo spraviť kĺzavý priemer. Toto riešenie malo tiež svoje chyby a~to v~tom, že~zmeny
zaznamenaných impulzov za sekundu sa zmenili v~závislosti od~rýchlosti a~smeru Obr.~\ref{fig:rw_lw_nf}. V~tomto bode sme vyskúšali počítať odometriu
z~obdržaných dát. Táto implementácia bola veľmi nepresná. Zároveň nám tento pokus potvrdil, že~potrebujeme filtrovať dáta, ktoré dostávame
od~robota, a~ktoré reprezentujú jeho rýchlosť v~impulzoch. Výsledky pokusu, kde sme zisťovali prevodovú charakteristiku z~impulzov za sekundu na~rýchlosť
v~SI jednotkách nájdeme na~nasledovných grafoch.

\begin{figure}[!htbp]
	\begin{subfigure}{0.5\textwidth}
		\includegraphics[width=\textwidth]{img/lw_nf.png}
	\end{subfigure}
	\hfill
	\begin{subfigure}{0.5\textwidth}
		\includegraphics[width=\textwidth]{img/rw_nf.png}
	\end{subfigure}
	\caption{Získanie prevodu z~impulzov na~rýchlosť v~SI jednotkách.}
	\label{fig:rw_lw_nf}
\end{figure}

Ako prvú vec sme si vykreslili všetky \textbf{nazbierané dáta}. Tie sú zobrazené \textbf{modrou} farbou. Cez ne sme spravili \textbf{lineárnu regresiu}. Je zobrazená
ako \textbf{červená} úsečka. Z~nazbieraných dát sme si nakoniec spravili priemer, aby sme videli, ako presne aproximuje nami vypočítaná lineárna regresia priemer
nazbieraných dát. \textbf{Priemery} jednotlivých rýchlostí sú zobrazené ako \textbf{oranžové} body.

Môžeme si všimnúť, že~vypočítané priemery takmer presne ležia na~vypočítanej lineárnej regresii. Problémom je, ako už bolo spomenuté, že~dáta,
ktoré dostávame od robota sú veľmi zašumené. Preto z~nich nevieme priamo počítať polohu robota. Na zobrazenie veľkosti odchýlky sme spravili
meranie. Nechali sme robot aby prešiel dráhu štvorca so~stranou dlhou 1 meter a~rýchlosťou 0,5 \(\frac{m}{s}\). Výsledok bol veľmi nepresný.
Metrom sme si odmerali jeho \textit{x}-ovú a~\textit{y}-ovú súradnicu s~počiatkom v~bode, kde sme na~robote spustili náš ovládač. Jeho skutočná
poloha bola v~bode~(-0,3m,~0m). Čo~nám ale vypočítala odometriu je, že~sa robot nachádzal 4 metre od~počiatku súradnicového systému.

\subsection{Zisťovanie parametru \(\alpha\)}

Implementovali sme si preto dolnopriepustný kvadraticky filter. Fungovanie tohto filtra spočíva v~skombinovaní nového vstupného parametra a~starého parametra
uloženého vo~filtri v~danom pomere. Tento pomer je daný parametrom \textbf{alpha} $\alpha$.

$$ stavFiltra = \alpha * stavFiltra + (1 - \alpha) * nováVzorka $$

Jeho parameter \(\alpha\) sme získali viacerými meraniami. Začali sme najsilnejším filtrom s~hodnotou $\alpha$ rovnou  0,9.

\begin{figure}[!htbp]
	\begin{subfigure}{0.5\textwidth}
		\includegraphics[width=\textwidth]{img/lw_09250.png}
	\end{subfigure}
	\hfill
	\begin{subfigure}{0.5\textwidth}
		\includegraphics[width=\textwidth]{img/rw_09250.png}
	\end{subfigure}
	\caption{Získanie prevodu z~impulzov na~rýchlosť v~SI jednotkách. $\alpha$ = 0,9.}
	\label{fig:rw_lw_09250}
\end{figure}

Ako môžeme vidieť na~obrázkoch Obr.~\ref{fig:rw_lw_09250} a~Obr.~\ref{fig:rw_lw_nf} aplikácia filtra výrazne pomohla proti~šumu signálu.
Problémom pri~silnom filtri je to, že~ak na~začiatku merania dostaneme zlú hodnotu, tak~sa táto hodnota ťažko mení na~správnu. Tento
efekt si môžeme všimnúť skoro pri~každej meranej rýchlosti. Najviditeľnejší dopad môžeme vidieť pri~pravom kolese
na~Obr.~\ref{fig:rw_lw_09250} pri~rýchlosti -0,6$\frac{m}{s}$. Tento problém sme riešili postupným menením parametra filtru.
Aby sme predišli veľkému množstvu meraní, tak~sme použili metódy binárneho vyhľadávania. V~tomto prípade sme začali s~veľkou hodnotou
a~postupne sme skákali do~stredu nášho intervalu. Preto sme si ako ďalšiu hodnotu zvolili alphu $\alpha$ rovnú 0,7.

\begin{figure}[!htbp]
	\begin{subfigure}{0.5\textwidth}
		\includegraphics[width=\textwidth]{img/lw_07250.png}
	\end{subfigure}
	\hfill
	\begin{subfigure}{0.5\textwidth}
		\includegraphics[width=\textwidth]{img/rw_07250.png}
	\end{subfigure}
	\caption{Získanie prevodu z~impulzov na~rýchlosť v~SI jednotkách. $\alpha$ = 0,7.}
	\label{fig:rw_lw_07250}
\end{figure}

Pri~použití hodnoty $\alpha$ rovnou 0,7 (Obr.~\ref{fig:rw_lw_07250}) sme zistili, že~sa hodnoty rýchlosti aj pri~aplikácii filtra výrazne menili. Ustálené
hodnoty zobrazené oranžovou farbou sú podobne ako pri~meraní s~filtrom s~alphou $\alpha$ rovnou 0,9 mimo lineárnej regresie.
Je to zapríčinené iným dôvodom ako pri~silnejšom filtri. Pokým pri~silnejšom filtri sme dostali zlú začiatočnú hodnotu, tak
už bolo zložité ju zmeniť. Pri~slabšom filtri, ak dostávame rozdielne vstupné hodnoty tak~sa výstupná hodnota filtra ľahko
mení. To má za dôsledok posun priemeru vstupných hodnôt. Tento efekt sa dá odstrániť zosilnením filtra, čiže zväčšením
koeficientu alpha $\alpha$. Spravili sme preto ďalšie meranie, kde sme použili hodnotu $\alpha$ rovnou 0,75.

\begin{figure}[!htbp]
	\begin{subfigure}{0.5\textwidth}
		\includegraphics[width=\textwidth]{img/lw_075250.png}
	\end{subfigure}
	\hfill
	\begin{subfigure}{0.5\textwidth}
		\includegraphics[width=\textwidth]{img/rw_075250.png}
	\end{subfigure}
	\caption{Získanie prevodu z~impulzov na~rýchlosť v~SI jednotkách. \(\alpha\) = 0,75.}
	\label{fig:rw_lw_075250}
\end{figure}

Obr.~\ref{fig:rw_lw_075250} zobrazuje graf výsledkov nameraných pri~aplikácii filtra s~koeficientom alpha o~veľkosti 0,75.
V~tomto prípade sa priemerné hodnoty na~rozdiel od~filtrov s~koeficientami alpha 0,9 a~0,7 dostali takmer priamo na~úsečku
lineárnej regresie prevodu z~impulzov za~sekundu na~metre za~sekundu. Použitie silnejšieho filtra síce pomohlo rýchlejšiemu
ustáleniu hodnoty, ale skúsili sme ešte silnejší filter s~hodnotou alpha $\alpha$ rovnou 0,8.

\begin{figure}[!htbp]
	\begin{subfigure}{0.5\textwidth}
		\includegraphics[width=\textwidth]{img/lw_08250.png}
	\end{subfigure}
	\hfill
	\begin{subfigure}{0.5\textwidth}
		\includegraphics[width=\textwidth]{img/rw_08250.png}
	\end{subfigure}
	\caption{Získanie prevodu z~impulzov na~rýchlosť v~SI jednotkách. $\alpha$ = 0,8.}
	\label{fig:rw_lw_08250}
\end{figure}

Ako môžeme vidieť na~Obr.~\ref{fig:rw_lw_08250} ustálenie hodnôt je veľmi jednoznačné. Vybrali sme si preto filter s~hodnotou
koeficientu alpha $\alpha$ rovnou 0,8. Zatiaľ všetky dáta čo sme merali boli s~frekvenciou 4Hz (1 vzorka za 250 milisekúnd).
Pre~presnejší výsledok sme túto frekvenciu ešte zvýšili. Z~testov robota sme vypozorovali, že~najfrekventovaniejsia frekvencia,
ktorú môže robot sprostredkovať je 10Hz (1 vzorka za 100 milisekúnd). Spravili sme si preto test na~prevod rýchlosti ešte raz
s~rovnakou hodnotou koeficientu alpha $\alpha$ rovnou 0,8, ale s~frekvenciou 10Hz namiesto už spomenutých 4Hz.

\begin{figure}[!htbp]
	\begin{subfigure}{0.5\textwidth}
		\includegraphics[width=\textwidth]{img/lw_08100.png}
	\end{subfigure}
	\hfill
	\begin{subfigure}{0.5\textwidth}
		\includegraphics[width=\textwidth]{img/rw_08100.png}
	\end{subfigure}
	\caption{Získanie prevodu z~impulzov na~rýchlosť v~SI jednotkách. $\alpha$ = 0,8 a~frekvenciou 10Hz.}
	\label{fig:rw_lw_08100}
\end{figure}

Najlepšia možná hodnota tohto koeficientu nám vyšla $\alpha$ rovná 0,8. Pri~implementácii tohto filtra sme museli myslieť na~dôležitú vec.
Keď sa zmenia jednorazovo impulzy na~hodnotu 0 a~hneď spať, tak~nám táto vzorka pokazí výsledok. Musíme preto túto vzorku ignorovať. Ďalšia
prekážka, ktorú sme mali pred sebou bola zmena rýchlosti. Obyčajná implementácia filtra by nám spomalila zmenu vypočítanej rýchlosti
a~teda aj veľkú odchýlku v~polohe. Tento problém sme opravili prestavením počiatočnej hodnoty filtra na~prvú hodnotu po~zmene rýchlosti.
Toto riešenie sa ukázalo ako najlepšie so skúšaných riešení.

Problém so~zlou počiatočnou hodnotou môžeme vidieť aj na~posledom grafe Obr.~\ref{fig:rw_lw_08100_3}. Tento problém sme vyriešili
predpočítavaním prvej hodnoty filtra po~zmene rýchlosti. Keďže sme už mali koeficienty lineárnej regresie, tak~sme ich využili
na~predpočítanie počiatočnej hodnoty. Výsledok tohto postupu vidíme na~nasledujúcom grafe.

\begin{figure}[!htbp]
	\begin{subfigure}{0.5\textwidth}
		\includegraphics[width=\textwidth]{img/lw_08100_3.png}
	\end{subfigure}
	\hfill
	\begin{subfigure}{0.5\textwidth}
		\includegraphics[width=\textwidth]{img/rw_08100_3.png}
	\end{subfigure}
	\caption{Získanie prevodu z~impulzov na~rýchlosť v~SI jednotkách. $\alpha$ = 0,8 a~frekvenciou 10Hz a~prvou prepočítanou hodnotou.}
	\label{fig:rw_lw_08100_3}
\end{figure}

