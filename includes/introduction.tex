% !TEX root = ../main.tex

Robotizácia sa v~dnešnej dobe viac a~viac rozširuje k~bežným ľuďom a~do~ich príbytkov. Cieľom všetkých vedcov
a~inžinierov je vytvoriť roboty, ktoré budu schopné pomáhať ľuďom vo~vykonávaní každodenných činností. Tým,
že~lúdia žijú v~domoch a~bytoch o~rôznych veľkostiach, tak~je potrebné vytvoriť roboty, ktoré sa budu vedieť
pohybovať v~užších ale~aj~otvorených priestoroch. Keď sa pozrieme na~dnešné ľudom prístupné mobilné roboty,
tak~majú väčšinou diferenciálne podvozky, ktoré sú~schopné pohybovať sa po~rovine, na~ktorej sa nenachádzajú veľké
prekážky. To~sú~hlavne parkety, dlaždice alebo koberce. Tento trend sa dlhšiu dobu nezmení a~bude sa len posúvať
vpred.

V~súčasnej dobe je pri~vytváraní softvéru pre~robota často využívaný Robotický Operačný Systém (ROS1) prvej verzie.
Tento systém aktuálne prechádza zo~staršej už~spomenutej verzie ROS1 na~novšiu ROS2. Tá dáva programátorom viac
možností ako spraviť systém užitočnejší a~prístupnejší pre~každodenného užívateľa.

Na~základe tohto trendu bol vytvorený projekt na~sprístupnenie ovládania mobilného robota do~prostredia Robotického
operačného systému druhej verzie. Toto sprístupnenie ovládania robota otvára dvere mnohým ďalším nápadom a~projektom.
Vytvorenie tohto rozhrania poskytuje možnosť implementácie rozhrania pre~ovládanie robota pomocou hlasu, gestami alebo
autonómnymi algoritmami.

Na~to~aby sme vedeli využiť všetky tieto výhody, tak~je potrebné mať ovládač, ktorý bude vytvárať rozhranie medzi
robotickým operačným systémom a~komunikačným rozhraním robota. V~tejto práci sa budeme venovať implementácii viacerých
procesov, ktoré budu spracovávať informácie poslané robotickým operačným systémom. Tieto správy spracujeme
do~formátu JSON s~tým, že~prepočítame vstupné lineárne a~uhlové rýchlosti zo~správy tejto témy na~rýchlosti ľavého
a~pravého kolesa diferenciálneho podvozku. Zároveň výstupom bude vypočítaná odometria robota, ktorá obsahuje údaje
o~rýchlosti, smere a~polohe robota.
