\section{Pôvodný stav robota}

\subsection{Robot a~jeho ovládanie}

Robot, s~ktorým sme pracovali bol výsledkom tímového projektu viacerých študentov \newline z~roku 2019. Pri~vysvetľovaní a~opisovaní robota sa budeme
odvolávať na~dokumenty, stránky a~kód, ktorý napísali. Všetky tieto údaje si sprístupnené na~mobilnom robote v~záložke
\newline \texttt{\$(HOME)/Desktop/Blackmetal}~\cite{timovyProjekt}.

Robot je v~tvare kvádra. Jeho šírka je 60cm a~je vyzdvihnutý nad zem o~1.5cm. Nachádza sa na~kolesách o~polomere 8cm. Jeho kostra, až na~oceľové pláty,
ktoré držia robot, je spravená z~hliníku. Konkrétne z~hliníkových tyčí, ktoré sú pospájané plexisklovými plátmi. Jeho podobizeň vidíme na~nasledujúcom
obrázku.

\begin{figure}[!htbp]
	\begin{center}
		\includegraphics[width=0.95\textwidth]{img/robot.png}
	\end{center}
	\caption{Zobrazenie spodnej časti mobilného robota~\cite{timovyProjekt}}
	\label{fig:robot}
\end{figure}

\noindent Na~obrázku ďalej vidíme olemovanie robota pásom s~LED-kami. Tie svietia nasledovným spôsobom. Keď sa robot nehýbe všetky LED-ky svietia
na zeleno. Keď sa robot pohne do~nejakej strany, LED-ky znázornia jeho pohyb tým, že~svietia na~strane, do~ktorej sa robot hýbe. Keď nastane
situácia, kedy počítač ovládajúci motory prestane komunikovať s~Arduinom, ktoré sa stará o~detekciu stavov robota tak~LED-ky začnú blikať
červeno-modrými farbami.

Ako bolo spomenuté LED-ky znázorňujú pohyb robota. Ten sa pohybuje za pomoci diferenciálneho podvozku s~dvoma podpornými všesmerovými kolesami.
Motory robota sú pripojené na~meniče. Tie sú ovládané priamo príkazmi z~počítača.

Hardware robota sa skladá z:
\begin{itemize}
	\item kontrolnej dosky Arduino Uno,

	\item Počítača ADVANTECH MIO-5272~\cite{robotPc} \newline
		Počítač obsahuje operačný systém Ubuntu 16.04.

	\item Extension board MIOe-210~\cite{extensionModule}

	\item Meniče MAXON EPOS 24/5 (s číslom 275512)~\cite{menic} \newline
	 	Sú napájané jednosmerným napätím 11 - 24 V~a~5 A.

	\item Enkódery MAXON Encoder MR Type L (s číslom 225787)~\cite{encoder} \newline
		Rozlíšenie enkóderov je 1024 impulzov s~troma kanálmi.

	\item Motory MAXON RE 40 (s číslom 148867)~\cite{motor} \newline
		Motory s~výkonom 150W. Maximálna rýchlosť je 12 000 rpm a~efektivita 91\%.

	\item Prevodovka MAXON Planetary Gearhead GP 42 C (s číslom 202120)~\cite{prevodovka} \newline
		Redukcia prevodovky je 43:1. Jej účinnosť je 72\%.
\end{itemize}

\noindent Ovládanie robota je zabezpečené externými počítačmi
\begin{itemize}
	\item Control PC (Kontrolný počítač) -- Počítač posielajúci príkazy na~robot cez~TCP/IP protokol.
	\item Logging PC (Logovací počítač) -- Počítač prijímajúci stav robota cez~TCP/IP protokol.
\end{itemize}

\begin{figure}[!htbp]
	\begin{center}
		\includegraphics[width=9cm]{img/schemaRobota.png}
	\end{center}
	\caption{Schéma zapojenia jednotlivých častí na~robote}
	\label{fig:schemaRobota}
\end{figure}

\noindent Tieto počítače sú len reprezentácia servera. V~realite to môže byť jeden a~ten istý počítač.

\noindent Na obrázku Obr.~\ref{fig:schemaRobota} vidíme zapojenie jednotlivých častí robota. Čo sme nespomenuli a~je na~obrázku je XBox ovládač
je to~kvôli tomu, že~tímový projekt bol zameraný na~ovládanie robota pomocou tohto ovládača. My ho ale použivať nebudeme.

\subsection{Komunikácia s~robotom}

S robotom sa vieme spojiť pomocou dvoch portov. Jeden port je otvorený na~prijímanie požiadavok (requestov) a~ten druhý je na~monitorovanie
stavu robota. Port \textit{664} je otvorený pre~tisíc užívateľov, ktorí môžu len sledovať stav robota. Druhý port je na~prijímanie requestov
\textit{665} a~je otvorený len pre~jedného užívateľa.

\subsubsection{Logovanie}
\label{sec:logovanie}

	Spomínaný port \textit{664} je otvorený jednému užívateľovi. Keď sa užívateľ pripojí začne dostávať nepretržite správy typu
	\textit{JSON}~(\acrlong{JSON}), ktoré hlásia stav robota. Správy, ktoré dostávame sú nasledujúceho formátu

	\begin{lstlisting}
			{"state":1,"direction":1}
	\end{lstlisting}

	Hodnoty sa pri~stave (state) a~ani pri~smere (direction) nemenia. Sú to~stále jednotky. Pokým robota nezastavíme buď príkazom, stlačením
	tlačidla vypnutia alebo zablokovaním jedného z~kolies, tak~sa tieto správy budú posielať. Môžeme potom začať polemizovať o~tom či~by nebolo
	lepšie už tieto správy využiť na~to~čo reálne spomenutý \textit{JSON} reťazec ukazuje. A~to udávať smer a~stav robota. Momentálne tieto správy
	slučia len na~to, aby sme vedeli, že~tento robot je aktívny a~vie primať a~spracúvať informácie.

\subsubsection{Ovládanie}
\label{sec:ovladanie}

	Port \textit{665} je sprístupnený na~prijímanie a~odosielanie požiadavok a~ich odpovedí. Príkazy sa na~počítač posielajú cez~sieť z~externého
	počítača vo~formáte \textbf{JSON}. Študenti, ktorí navrhovali systém posielania požiadavok (request) a~odpovedí (response) robili tieto správy
	ručne. Preto~nastávajú situácie, kedy robot pošle správu, ktorá nespadá do~štandardu písania JSON textu. Z~tohto dôvodu sme nemohli použiť už
	existujúci kód (parser), ktorý by nám zjednodušil prehľadávanie týchto správ. Podla dokumentácie sa robot mal ovládať správami typu~\cite{BMdoc}

	\label{jsonSpeedRequestBad}
	\begin{lstlisting}
			{"UserID":1,"Command":3,"RightWheelSpeed":50,"LeftWheelSpeed":50}
	\end{lstlisting}

	\newpage

	\noindent Význam jednotlivých parametrov:
	\begin{itemize}
		\item \textbf{UserID} -- Znázorňuje ID užívateľa, ktorý je pripojený na~robot. Predvolená hodnota je 1.
		\item \textbf{Command} --  Číselná hodnota znázorňujúca príkaz, ktorý ma robot vykonať
			\begin{enumerate}
				\setcounter{enumi}{-1}
				\item \label{c0} Prázdny príkaz slúžiaci na~overenie spojenia
				\item \label{c1} Núdzové zastavenie
				\item \label{c2} Normálne zastavenie
				\item \label{c3} Príkaz nastavujúc rýchlosti kolies mobilného robota
				\item \label{c4} Prázdny príkaz
				\item \label{c5} Prázdny príkaz
				\item \label{c6} Príkaz pýtajúci si aktuálnu rýchlosti pravého a~ľavého kolesa. Tento príkaz nebol sprave navrhnutý v~kóde
					robota. Vracal nám žiadaná hodnotu namiesto aktuálnej. Museli sme ho prepísať.
				\item \label{c7} Pripravenie motorov robota
				\item \label{c8} Príkaz pýtajúci si aktuálnu pozíciu pravého a~ľavého kolesa.
			\end{enumerate}
		\item \textbf{RightWheelSpeed} -- Nastavenie rýchlosti pre~pravé koleso
		\item \textbf{LeftWheelSpeed} -- Nastavenie rýchlosti pre~ľavé koleso
	\end{itemize}

	\noindent Z~tohto kusu kódu je jasné, že~sa majú posielať celé čísla a~na základe tohto vstupu sa bude robot hýbať. Čo sme zistili až
	po skompilovaní a~spustení tímového projektu je, že~sa majú posielať desatinné čísla z~intervalu 0 až 1. Toto nebolo písané
	v~dokumentácii, ktorá nám bola dodaná na~začiatku programu. Môžeme preto príklad prepísať na~reťazec, ktorý by fungoval

	\label{jsonSpeedRequestGood}
	\begin{lstlisting}
			{"UserID":1,"Command":3,"RightWheelSpeed":0.50,"LeftWheelSpeed":0.50}
	\end{lstlisting}

\subsection{Vysvetlenie kľúčov reťazca}

\noindent \textbf{UserID} \newline
\indent Táto možnosť je v~momentálnom stave robota nevyužitá. Počet zariadení, ktoré sa môžu pripojiť na~port, cez~ktorý sa dá robot ovládať
je 1. Je to~ale dobrá možnosť na~rozšírenie kódu. Keď sa budú môcť pripojiť viacerí užívatelia, tak~sa bude musieť vyriešiť, koho príkaz
bude mať akú prioritu. \newline

\noindent \textbf{Command:~\ref{c4}} \newline
\indent Tento príkaz je prázdny. My sme ho ale neskôr prepísali na~príkaz, cez~ktorý sa dá nastaviť žiadaná pozícia kolies robota (natočenie).
Táto funkcionalita nie je v~takom stave ako sme si priali. Je to spôsobené hlavne nedostačujúcov dokumentáciou enkóderov na~robote. Síce sme našli
v~dokumentácii funkciu, ktorá by mala túto možnosť povoľovať. Čo sa ale stane pri~poslaní príkazu je to, že~kolesá sa začnú točiť rýchlosťou
0,5 metra za sekundu.\newline

\noindent \textbf{Command:~\ref{c8}} \newline
\indent Príkaz na~zisťovanie polohy kolies nebol originálne naprogramovaný na~robote. Pridali sme ho za cieľom presného dostavia sa robota
na~preddefinované miesto. Táto funkcionalita nefunguje správne rovnako ako v~predchádzajúcom príklade, keď si vypýtame polohu kolies od robota,
dáta ktoré obdržme sú, že~jedno koleso je priamo nastavené na~hodnotu, ktorú sme si vyžiadali a~to druhé koleso vráti náhodnú hodnotu.
Počas toho sa ale kolesá robota stále točia.\newline

\noindent \textbf{RightWheelSpeed/LeftWheelSpeed} \newline
\indent Nastavovanie rýchlosti pravého a~ľavého kolesa nie sú povinné parametre. Musíme ich zadávať len v~prípade posielania rýchlostí
cez~príkaz s~číslom \ref{c3} alebo \ref{c4}.
