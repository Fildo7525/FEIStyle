% !TEX root = ../main.tex
This bachelor's thesis deals with the control of a mobile robot in the ROS2 environment. The robot we will
be working with is equipped with a differential chassis and two support omnidirectional wheels. It contains
two encoders for the right and left wheel, which measure the wheel velocities in pulses per second.
The control and communication interface with the robot is implemented via TCP/IP connection. Messages
in JSON structure are sent when communicating with the robot.
The ultimate goal of this bachelor's thesis is to create a robot controller that will provide access
to robot control in the ROS2 environment. In the first chapter, we analyse the original state
of the robot and its communication interface. The second chapter describes the design of the robot
controller implementation in the ROS2 environment. We devoted the third chapter to describing
the improvements we made to the robot. The fourth chapter focuses on the actual implementation
of the robot controller. In the penultimate chapter, we describe how we overcame issues that arose
during communication with the robot. In the last chapter, we describe several uses of the robot
controller in the ROS2 environment.
